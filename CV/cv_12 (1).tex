%%%%%%%%%%%%%%%%%%%%%%%%%%%%%%%%%%%%%%%
% Deedy CV/Resume
% XeLaTeX Template
% Version 1.0 (5/5/2014)
%
% This template has been downloaded from:
% http://www.LaTeXTemplates.com
%
% Original author:
% Debarghya Das (http://www.debarghyadas.com)
% With extensive modifications by:
% Vel (vel@latextemplates.com)
%
% License:
% CC BY-NC-SA 3.0 (http://creativecommons.org/licenses/by-nc-sa/3.0/)
%
% Important notes:
% This template needs to be compiled with XeLaTeX.
%
%%%%%%%%%%%%%%%%%%%%%%%%%%%%%%%%%%%%%%

\documentclass[letterpaper]{deedy-resume} % Use US Letter paper, change to a4paper for A4 
\usepackage{graphics}
\begin{document}

%----------------------------------------------------------------------------------------
%	TITLE SECTION
%----------------------------------------------------------------------------------------

\lastupdated % Print the Last Updated text at the top right

\namesection{Federica}{Di Stefano}{ % Your name
\urlstyle{same}\url{https://logic-cs.at/phd/students/federica-di-stefano/} \\ % Your website, LinkedIn profile or other web address
\href{federica.stefano@tuwien.ac.at}{fede12distefano@gmail.com}% Your contact information
}

%----------------------------------------------------------------------------------------
%	LEFT COLUMN
%----------------------------------------------------------------------------------------

\begin{minipage}[t]{0.33\textwidth} % The left column takes up 33% of the text width of the page

%------------------------------------------------
% Education
%------------------------------------------------

\section{Education} 

\subsection{TU Wien}

\descript{PhD in Computer Science}
\location{Expected Oct 2024 | Vienna, Austria}

\sectionspace

\subsection{Università degli Studi di Salerno}

\descript{MSc in Mathematics}
\location{Febr 2019 | Salerno, Italy \\ Final grade: 110/110 cum laude}
Title of the thesis: \emph{A strong complete semantics for {\L}ukasiewicz Logic}\\
Supervisor: \emph{Luca Spada}
\sectionspace % Some whitespace after the section

\descript{BSc in Mathematics}
\location{Dec 2016 | Salerno, Italy \\ Final grade: 110/110 cum laude}
Title of the thesis: \emph{Geometria senza punti: approcci metrici e dualità logiche}
Supervisor: \emph{Cristina Coppola}
\sectionspace % Some whitespace after the section

%------------------------------------------------

\subsection{Liceo Scientifico "A. Romita"}

\location{Grad. July 2013 | Campobasso, Italy}

\sectionspace % Some whitespace after the section

%------------------------------------------------
% Links
%------------------------------------------------

%pace after the section

%------------------------------------------------
% Coursework
%------------------------------------------------
\descript{Attended Summer Schools}
\sectionspace
\begin{tightitemize}
\item Reasoning Web Summer School
\item ESSLLI2021
\item AILA Summer School
\end{tightitemize}
\sectionspace
\section{Links} 

%Github:// \href{https://github.com/deedydas}{\bf deedydas} \\
LinkedIn:// \href{https://www.linkedin.com/in/federica-di-stefano-a04673194/}{\bf Federica Di Stefano} \\

\sectionspace % Some whites
\section{Language Skills}
Italian - Mother Tongue \\
English - 6.5 IELTS\\
German - A2 (not certificated) 
\sectionspace
\section{Hobbies}
Astronomy, 
Lego\textregistered, 
Board Games of any type,
Marvel Comics \& Movies.


%\subsection{Graduate}
%
%Advanced Machine Learning \\
%Open Source Software Engineering \\
%Advanced Interactive Graphics \\
%Compilers + Practicum \\
%Cloud Computing
%
%\sectionspace % Some whitespace after the section
%
%%------------------------------------------------
%
%\subsection{Undergraduate}
%
%Information Retrieval \\
%Operating Systems \\
%Artificial Intelligence + Practicum \\
%Functional Programming \\
%Computer Graphics + Practicum \\
%{\footnotesize \textit{\textbf{(Research Asst. \& Teaching Asst) }}} \\
%Unix Tools and Scripting
%
%\sectionspace % Some whitespace after the section
%
%%------------------------------------------------
%% Skills
%%------------------------------------------------
%
%\section{Skills}
%
%\subsection{Programming}
%
%\location{Over 5000 lines:}
%Java \textbullet{} Shell \textbullet{} JavaScript \textbullet{} Matlab \\
%OCaml \textbullet{} Python \textbullet{} Rails \textbullet{} \LaTeX\ \\ 
%\location{Over 1000 lines:}
%C \textbullet{} C++ \textbullet{} CSS \textbullet{} PHP \textbullet{} Assembly \\
%\location{Familiar:}
%AS3 \textbullet{} iOS \textbullet{} Android \textbullet{} MySQL
%
%\sectionspace % Some whitespace after the section
%
%%----------------------------------------------------------------------------------------
%
\end{minipage} % The end of the left column
\hfill
%
%----------------------------------------------------------------------------------------
%	RIGHT COLUMN
%----------------------------------------------------------------------------------------
%
\begin{minipage}[t]{0.66\textwidth} % The right column takes up 66% of the text width of the page

%------------------------------------------------
% Experience
%------------------------------------------------

\section{Experience}

\runsubsection{TU Wien}
\descript{| Project Assistant in the Databases and Artificial Intelligence Group}

\location{Since October 2020 | TU Wien, Vienna}
\vspace{\topsep} % Hacky fix for awkward extra vertical space
\begin{tightitemize}
\item I am part of the Doctoral School `Logics in Computer Science'. As a PhD student, I am a Project Assistant with the FWF Project `KtoAPP: Compiling Knowledge into Applications' under the supervision of Mantas Simkus and the co-supervision of Magdalena Ortiz. My current work is located in the field of Theoretical Computer Science. We work with Description Logics, a family of languages for Knowledge Representation kindred to first-order logic. In particular, we work on non-monotonic extensions of Description Logics based on Circumscription. 
\end{tightitemize}

\sectionspace % Some whitespace after the section

%------------------------------------------------
\runsubsection{Università degli Studi di Salerno}
\descript{| Tutor in Topology}

\location{Mar 2020 – June 2020 | Salerno, Italy}
\begin{tightitemize}
\item I worked as tutor for the mandatory course `Geometry III' for the BSc in Mathematics. 
\end{tightitemize}

\sectionspace

\runsubsection{Università degli Studi di Salerno}
\descript{| Research Scholarship}

\location{Febr 2020 – Sept 2020 | Salerno, Italy}
\begin{tightitemize}
\item We worked on the semantics of {\L}ukasiewcz Logic, one of the prominent many-valued logics. In particular, we worked on the geometric characterization of ideals of MV-algebras. 
\end{tightitemize}

\sectionspace % Some whitespace after the section

%------------------------------------------------

\runsubsection{EurekaApprendimento}
\descript{| Tutor in Mathematics}

\location{Sept 2019 – Jan 2020 | Salerno, Italy}
\begin{tightitemize}
\item As tutor in Mathematics, my work was focused on supporting the learning process of high-school students with difficulties in Mathematics. 
\end{tightitemize}

\sectionspace
\runsubsection{Liceo Classico `De Sanctis' di Salerno }
\descript{| Tutor}
\location{Mar 2019 - Jun 2019 | Salerno, Italy}
\begin{tightitemize}
\item I worked as Tutor in Mathematics for high school students in the project `MATAID'. 
\end{tightitemize}
\sectionspace

\runsubsection{Università degli Studi di Salerno}
\descript{| Tutor in Mathematics}

\location{Oct 2017 – Febr 2018 | Salerno, Italy}
\begin{tightitemize}
\item I worked as tutor for the course `Mathematics I' for the BSc in Environment Science. 
\end{tightitemize}

\sectionspace


\runsubsection{Stage at Town Hall of Pietracatella} 
\descript{July 2013- Oct 2013 | Pietracatella, Campobasso, Italy}
\begin{tightitemize}
\item Stage offered by the Italian Region Molise to support the integration of graduated students in the workforce. 
\end{tightitemize}

\sectionspace % Some whitespace after the section

%------------------------------------------------
% Research
%------------------------------------------------

%\section{Research}
%
%\runsubsection{Cornell Robot Learning Lab}
%\descript{| Head Undergrad Research}
%
%\location{Jan 2014 – Present | Ithaca, NY}
%Worked with \textbf{\href{http://www.cs.cornell.edu/~ashesh/}{Ashesh Jain}} and \textbf{\href{http://www.cs.cornell.edu/~asaxena/}{Prof Ashutosh Saxena}} to create \textbf{PlanIt}, a tool which learns from large scale user preference feedback to plan robot trajectories in human environments. Publication submitted.
%
%\sectionspace % Some whitespace after the section
%
%%------------------------------------------------
%
%\runsubsection{Cornell Phonetics Lab}
%\descript{| Head Undergraduate Researcher}
%
%\location{Mar 2012 – May 2013 | Ithaca, NY}
%Lead the development of \textbf{QuickTongue}, the first ever breakthrough tongue-controlled game with \textbf{\href{http://conf.ling.cornell.edu/~tilsen/}{Prof Sam Tilsen}} to aid in Linguistics research. Publication submitted.

%\sectionspace % Some whitespace after the section

%------------------------------------------------
% Awards
%------------------------------------------------

\section{Awards} 

\begin{tabular}{rll}
2013-2018 & Merit Scholarship awarded by the Italian Ministry of Education\\
2014-2017 & Scholarship awarded by A.D.I.S.U. for deserving students
\end{tabular}

\sectionspace % Some whitespace after the section

%------------------------------------------------
% Societies
%------------------------------------------------
\end{minipage} % The end of the right column

%----------------------------------------------------------------------------------------
%	SECOND PAGE (EXAMPLE)
%----------------------------------------------------------------------------------------

\newpage % Start a new page

\begin{minipage}[t]{0.33\textwidth} % The left column takes up 33% of the text width of the page

\section{}

\end{minipage} % The end of the left column
\hfill
\begin{minipage}[t]{0.66\textwidth} % The right column takes up 66% of the text width of the page
\section{Publications} 
\descript{Proceedings with Referee} 
\sectionspace
\begin{tightitemize}
\item -, M. Ortiz, M. \v{S}imkus, \emph{Pointwise Circumscription in Description Logics}, DL2022, Haifa, Israel.
\item M. Abbadini, -, L. Spada, \emph{Unification in {\L}ukasiewcz Logic with a finite number of variables}, IPMU 2020, Online.
\end{tightitemize}
\descript{Informal Proceedings (with Referee)}
\begin{tightitemize}
\item -, \emph{Pointwise Circumscribed Description Logics: A Local Approach to Non-Monotonicity}, Doctoral Consortium - KR2021, Online.
\end{tightitemize}

\sectionspace % Some whitespace after the section

%----------------------------------------------------------------------------------------



\section{Research Activities} 
\descript{Research Visits}
\begin{tightitemize}
\item 23th-30th October 2022, visiting PhD Student at Ume\aa\, University, Sweden.
\item 16th-18th November 2021, visiting PhD Student at University of Naples, Italy. 
\end{tightitemize}
\descript{Presentations}
\begin{tightitemize}
\item DL2022 (in joint session with NMR2022). Talk: \emph{Pointwise Circumscription in Description Logics}. 
\item KR2021 Doctoral Consortium. Talk: \emph{Pointwise Circumscribed Description Logics: a local approach to non-monotonicity}.
\item IPMU2020. Talk: \emph{Unification in {\L}ukasiewicz Logic with a finite number of variables}.
\end{tightitemize}
%

\end{minipage} % The end of the right column

%----------------------------------------------------------------------------------------

\end{document}